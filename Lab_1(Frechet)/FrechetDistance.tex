\documentclass[a4]{article}
\pagestyle{myheadings}

%%%%%%%%%%%%%%%%%%%
% Packages/Macros %
%%%%%%%%%%%%%%%%%%%
\usepackage{mathrsfs}


\usepackage{fancyhdr}
\pagestyle{fancy}
\lhead{}
\chead{}
\rhead{}
\lfoot{}
\cfoot{} 
\rfoot{\normalsize\thepage}
\renewcommand{\headrulewidth}{0pt}
\renewcommand{\footrulewidth}{0pt}
\newcommand{\RomanNumeralCaps}[1]
    {\MakeUppercase{\romannumeral #1}}

\usepackage{amssymb,latexsym}  % Standard packages
\usepackage[utf8]{inputenc}
\usepackage[russian]{babel}
\usepackage{MnSymbol}
\usepackage{mathrsfs}
\usepackage{amsmath,amsthm}
\usepackage{indentfirst}
\usepackage{graphicx}%,vmargin}
\usepackage{graphicx}
\graphicspath{{pictures/}} 
\usepackage{verbatim}
\usepackage{color}
\usepackage[nottoc,numbib]{tocbibind}
\usepackage{float}

\usepackage{listings}
\definecolor{codegreen}{rgb}{0,0.6,0}
\definecolor{codegray}{rgb}{0.5,0.5,0.5}
\definecolor{codepurple}{rgb}{0.58,0,0.82}
\definecolor{backcolour}{rgb}{0.95,0.95,0.92}
 
\lstdefinestyle{mystyle}{
    backgroundcolor=\color{backcolour},   
    commentstyle=\color{codegreen},
    keywordstyle=\color{magenta},
    numberstyle=\tiny\color{codegray},
    stringstyle=\color{codepurple},
    basicstyle=\footnotesize,
    breakatwhitespace=false,         
    breaklines=true,                 
    captionpos=b,                    
    keepspaces=true,                 
    numbers=left,                    
    numbersep=5pt,                  
    showspaces=false,                
    showstringspaces=false,
    showtabs=false,                  
    tabsize=2
}
 
\lstset{style=mystyle}

\usepackage{url}
\urldef\myurl\url{foo%.com}





\DeclareGraphicsExtensions{.pdf,.png,.jpg}% -- настройка картинок

\usepackage{epigraph} %%% to make inspirational quotes.
\usepackage[all]{xy} %for XyPic'a
\usepackage{color} 
\usepackage{amscd} %для коммутативных диграмм
%\usepackage[colorlinks,urlcolor=red]{hyperref}

%\renewcommand{\baselinestretch}{1.5}
%\sloppy
%\usepackage{listings}
%\lstset{numbers=left}
%\setmarginsrb{2cm}{1.5cm}{1cm}{1.5cm}{0pt}{0mm}{0pt}{13mm}


\newtheorem{Lemma}{Лемма}[section]
\newtheorem{Proposition}{Предложение}[section]
\newtheorem{Theorem}{Теорема}[section]
\newtheorem{Corollary}{Следствие}[section]
\newtheorem{Remark}{Замечание}[section]
\newtheorem{Definition}{Определение}[section]
\newtheorem{Designations}{Обозначение}[section]




%%%%%%%%%%%%%%%%%%%%%%% 
%Подготовка оглавления% 
%%%%%%%%%%%%%%%%%%%%%%% 
\usepackage[titles]{tocloft}
\renewcommand{\cftdotsep}{2} %частота точек
\renewcommand\cftsecleader{\cftdotfill{\cftdotsep}}
\renewcommand{\cfttoctitlefont}{\hspace{0.38\textwidth} \LARGE\bfseries} 
\renewcommand{\cftsecaftersnum}{.}
\renewcommand{\cftsubsecaftersnum}{.}
\renewcommand{\cftbeforetoctitleskip}{-1em} 
\renewcommand{\cftaftertoctitle}{\mbox{}\hfill \\ \mbox{}\hfill{\footnotesize Стр.}\vspace{-0.5em}} 
%\renewcommand{\cftchapfont}{\normalsize\bfseries \MakeUppercase{\chaptername} } 
%\renewcommand{\cftsecfont}{\hspace{1pt}} 
\renewcommand{\cftsubsecfont}{\hspace{1pt}} 
%\renewcommand{\cftbeforechapskip}{1em} 
\renewcommand{\cftparskip}{3mm} %определяет величину отступа в оглавлении
\setcounter{tocdepth}{5} 
\renewcommand{\listoffigures}{\begingroup %добавляем номер в список иллюстраций
\tocsection
\tocfile{\listfigurename}{lof}
\endgroup}
\renewcommand{\listoftables}{\begingroup %добавляем номер в список иллюстраций
\tocsection
\tocfile{\listtablename}{lot}
\endgroup}


   
   
%\renewcommand{\thelikesection}{(\roman{likesection})}
%%%%%%%%%%%
% Margins %
%%%%%%%%%%%
\addtolength{\textwidth}{0.7in}
\textheight=630pt
\addtolength{\evensidemargin}{-0.4in}
\addtolength{\oddsidemargin}{-0.4in}
\addtolength{\topmargin}{-0.4in}

%%%%%%%%%%%%%%%%%%%%%%%%%%%%%%%%%%%
%%%%%%Переопределение chapter%%%%%% 
%%%%%%%%%%%%%%%%%%%%%%%%%%%%%%%%%%%
\newcommand{\empline}{\mbox{}\newline} 
\newcommand{\likechapterheading}[1]{ 
\begin{center} 
\textbf{\MakeUppercase{#1}} 
\end{center} 
\empline} 

%%%%%%%Запиливание переопределённого chapter в оглавление%%%%%% 
\makeatletter 
\renewcommand{\@dotsep}{2} 
\newcommand{\l@likechapter}[2]{{\bfseries\@dottedtocline{0}{0pt}{0pt}{#1}{#2}}} 
\makeatother 
\newcommand{\likechapter}[1]{ 
\likechapterheading{#1} 
\addcontentsline{toc}{likechapter}{\MakeUppercase{#1}}} 




\usepackage{xcolor}
\usepackage{hyperref}
\definecolor{linkcolor}{HTML}{000000} % цвет ссылок
\definecolor{urlcolor}{HTML}{AA1622} % цвет гиперссылок
 
\hypersetup{pdfstartview=FitH,  linkcolor=linkcolor,urlcolor=urlcolor, colorlinks=true}

%%%%%%%%%%%%
% Document %
%%%%%%%%%%%%

%%%%%%%%%%%%%%%%%%%%%%%%%%%%%
%%%%%%главы -- section*%%%%%%
%%%%section -- subsection%%%%
%subsection -- subsubsection%
%%%%%%%%%%%%%%%%%%%%%%%%%%%%%
\def \newstr {\medskip \par \noindent} 



\begin{document}
\def\contentsname{\LARGE{Содержание}}
\thispagestyle{empty}
\begin{center} 
\vspace{2cm} 
{\Large \sc Санкт-Петербургский Политехнический}\\
\vspace{2mm}
{\Large \sc Университет} им. {\Large\sc Петра Великого}\\
\vspace{1cm}
{\large \sc Институт прикладной математики и механики\\ 
\vspace{0.5mm}
\textsc{}}\\ 
\vspace{0.5mm}
{\large\sc Кафедра прикладной математики}\\
\vspace{15mm}
%\rule[0.5ex]{\linewidth}{2pt}\vspace*{-\baselineskip}\vspace*{3.2pt} 
%\rule[0.5ex]{\linewidth}{1pt}\\[\baselineskip] 
{\huge \sc Лабораторная работа №$1$\\
\vspace{4mm}
Расстояние Фреше
\vspace{6mm}
 }
\vspace*{2mm}
%\rule[0.7ex]{\linewidth}{1pt}\vspace*{-\baselineskip}\vspace{3.2pt} 
%\rule[0.5ex]{\linewidth}{2pt}\\ 
\vspace{1cm}

{\sc $4$ курс$,$ группа $43631/2$}

\vspace{2cm} 
Студент \hfill Д. А. Плаксин\\
\vspace{1cm}
Преподаватель \hfill Баженов А. Н.\\
\vspace{20mm} 

\end{center} 
%\author{Я}
\begin{center}
\vfill {\large\textsc{Санкт-Петербург}}\\ 
2019 г.
\end{center}

%%%%%%%%%%%%%%%%%%%%%%%%%%%%%%%%%%%%%%%%%%%%%%%%%%%%%%%%%%%%%%%%%%%%%%%%%%%%%%%%%%%%%%%%%%%%%%
%\ \\[4cm]

%\rm
%%%%%%%%%%%%%%%%%%%%%%%%%%%%%%%%%%%%%%%%%%%%%%%%%%%%%%%%%%%%%%%%%%%%%%%%%%%%%%%%%%%%%%%%%%%%%%
\newpage
\pagestyle{plain}

%\begin{center}
%\begin{abstract} 

%\end{abstract}

%\end{center}
\newpage
\tableofcontents{}
\newpage
\listoffigures{}
\newpage

\section{Постановка задачи}

В данной задаче требуется построить ломанные кривые, реализовать вычисление расстояния Фреше между двумя ломанными и найти элементы точки, на которых вычислено это расстояние, с обоснованием точности результата и проверкой единственности.

\section{Теория}
В ходе решения некоторых математических задач возникает потребность в геометрической оценке свойств полученных множеств. Такой количественной оценкой может служить мера сходства форм областей.

Рассмотрим метрическое пространство с заданной на нём метрикой – $(V,d)$
Стандартный подход к вычислению расстояния Фреше между кривыми – вычисление дискретного расстояния Фреше для ломаных, которые приближают исходные кривые.

Стандартный подход к вычислению расстояния Фреше между кривыми – вычисление дискретного расстояния Фреше для ломаных, которые приближают исходные кривые.

Пусть $P:[0,n]\to V$ ломаная кривая; $Q\; \--$ ломаная кривая.  $L\; \--$ сопряжение между двумя кривыми.

Тогда дискретное расстояние Фреше:
\begin{equation}
    \delta_{dF}(P,Q)=min\|L\|
\end{equation}


\section{Реализация}
Для генерации выборки был использован $Python\;3.7$. Использовалась библиотека numpy, графики строились с помощью библиотеки $matplotlib.$


\newpage
\section{Результаты}

Для кривых:
$$P = [[0, 0], [4, 2], [6, 5], [12, 6], [15, 7], [15, 10], [18, 13]]$$
$$Q = [[1, 1], [2, 5], [7, 7], [8, 12], [13, 14], [15, 16]]$$

Ответ: $\delta_{dF}(P,Q) = 7.280109889280518$ между точками $(15,7)$ и $(13,14)$

\begin{figure}[H]
\caption{Расстояние Фреше для двух кривых }
\includegraphics[width=\textwidth]{fig1.png} 
\end{figure}
\newpage

Для двух замкнутых кривых, ограничивающих невыпуклые множества:
$$P = [[2, 2], [3, 4], [2, 7], [5, 6], [9, 8], [8, 5], [10, 1], [6, 3], [2, 2]]$$
$$Q = [[12, 1], [10, 3], [6, 6], [9, 7], [10, 9], [12, 6], [15, 5], [13, 3], [12, 1]]$$

Ответ: $\delta_{dF}(P,Q) = 10.04987562112089$.

\begin{figure}[H]
\caption{Расстояние Фреше для замкнутых кривых. }
\includegraphics[width=\textwidth]{fig2.png} 
\end{figure}

\section{Обсуждение}
\subsection{Точность результатов}
Библиотека $numpy$ оперирует числами с плавающей запятой двойной точности $\--$ $float64$, таким образом поддерживается точность до $15-17$ знака после запятой.

Из-за вычисления нормы в ходе расчётов точность снижается в два раза, так как алгоритм включает в себя вычисление нормы, содержащей операцию перемножения двух чисел $float64$ и составляет $7$ знаков после запятой.

\subsection{Единственность}
Для проверки единственности решения достаточно проверить на каждом шаге вычислений: \begin{equation}
dist = \delta_{dF}(P[i],Q[j]) = \max\left(\min(c[i-1,j], c[i-1,j-1],c[i,j-1]  ,\delta_{dF}(P[i],Q[j])\right)
\end{equation}
Тогда значение в $[i,j]$ оказывается равным расстоянию в точках $P[i], Q[j].$ Значит, решение не является единственным и расстояние Фреше можно найти между двумя другими точками.

\subsection{Трудоёмкость}
Алгоритм вычисления расстояния Фреше является рекурсивным, и наибольшую трудоёмкость имеют операции вычисления следующего выражения, которое часто вычисляется:

\begin{equation}
    \max\left(\min\left(d\left(a_{k_{i-1}},b_{m_j}\right),d(\left(a_{k_{i-1}},b_{m_{j-1}}\right),d\left(a_{k_i},b_{m_{j-1}}\right),d\left(a_{k_i},b_{m_j}\right)\right)\right)
\end{equation}

\begin{thebibliography}{}
    \bibitem{numpy}  Модуль numpy  -  https://physics.susu.ru/vorontsov/language/numpy.html
    
    \bibitem{plotlib} 
    Модуль matplotlib - https://matplotlib.org/users/index.html
    
    \bibitem{source}
    Пособие к Лабораторным работам https://cloud.mail.ru/public/4ra6/5wwqBzMBC/LabPractics.pdf
\end{thebibliography}

\section{Приложения}

Код отчёта:\; \url{https://github.com/MisterProper9000/computing-complexes-labs/blob/frechet-distance/frechet.tex}

Код лаборатрной:\; \url{https://github.com/MisterProper9000/computing-complexes-labs/blob/frechet-distance/frechet.py}

\lstinputlisting[language=Python]{frechet.py}

\end{document}